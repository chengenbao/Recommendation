% !Mode:: "TeX:UTF-8"
\documentclass[a4paper]{article}
\usepackage{graphicx}
\usepackage{titletoc}
\usepackage{titlesec}
%\usepackage{abstract}
\usepackage{fontspec, xunicode, xltxtra}
\usepackage{xeCJK}
\setmainfont{Times New Roman}
\setCJKmainfont{华文楷体}
\begin{document}
\renewcommand{\today}{\number\year 年 \number\month 月 \number\day 日}
\renewcommand{\contentsname}{目 \quad 录}
\title{基于内容的论文推荐系统}
\author{陈根宝}
\date{\today}
\maketitle
\newpage
\tableofcontents
\newpage
\section{引言}
论文推荐旨在为闪记科技工作者用户提供论文推荐功能。论文推荐不同于一般的推荐,论文有很强文本特性,%
因此能够用基于内容的推荐(标题,关键字,摘要);其次论文的作者一般具有特定的领域属性,因此论文能%
按其作者进行领域聚合;再次,论文一般都有相互之间的引用关系,这种引用关系可以看做是作者对相关文献%
的推荐,这好似是网页之间的相互链接,因此pagerank算法这里很好用。
\section{算法设计}
\subsection{算法输入}
我们的算法是以用户提交的阅读记录为推荐算法的输入。这些关键字可以从用户使用闪记产品对论文进行标记的%
时候获取,闪记客户端能在用户使用的时候识别论文,并能从其中摘取论文的五元组 --- 标题、作者、概要%
、引用、会议期刊名称,并将这些信息存入后台数据库,然后将用户id和论文id通过接口推送给推荐引擎。%
推荐引擎从后台数据库中读取论文五元组信息进行相关推荐。
\subsection{算法输出}
推荐算法的输出是与输入的阅读文献高度相关的n(n >= 1)篇论文的信息,这个结果会以文献id的形式写入到用%
户推荐数据库中。
\subsection{算法设计}
正如引言中的论文推荐具有以上的一些特点,所以论文推荐的算法应该兼顾这些特点,因此我们将算法有以下%
几个影响因子:
\begin{itemize}
\item 论文内容,包含论文标题、概要、用户摘取内容以及关键字
\item 论文作者
\item 论文引用关系
\item 读者行为
\end{itemize}
下面我们将对这些影响因子一一进行分析,并给出量化的计算方式。
\subsection{论文内容}
论文内容,这里指的并不是论文的正文,而是泛指区别于论文的作者、引用文献之外的一切文本实体,在此我们%
特指有论文的标题、概要、关键字和用户摘录内容混合之后经过一定处理得到的文本内容。

在用文本的内容进行推荐的时候,如果将论文内容和库里所有的文献进行内容相似度计算则计算量是十分大的,%
几乎是不能忍受的,问了降低计算复杂度,我们论文内容中抽取关键词并以此检索(因此事先我们要用sphinx对%
文献库建立倒排索引),然后将获得的文献集合与论文内容进行内容相似度匹配。

因此在该步骤中关键的有两步,1)提取论文内容特征词;2)内容相似度计算方法,下面我们分别给予介绍。
\subsubsection{论文内容特征词提取}
特征词提取的方法很多,有基于词性分析的、基于TF\-IDF阈值、基于Bigram的等多种方法。我们初步设计会选用%
TD、-IDF阈值的方法,这种方法实现起来会比较简单。但是英文中词的区别性不如词组好,比如~cloud computing~作%
为一个词组的时候表示的含义就很特殊,如果拆分则区分度会大大降低,所以我们应该在算法中加入词组的识别,%
这样会显著提高推荐效果。关于词组的识别,我认为可以采用字典的方法,将很多论文的keyword收集并存入字典中。%
\subsubsection{内容相似度计算}
相似度计算比较简单,直接采用向量空间模型(vector space model)进行计算。
\subsection{论文作者}
\subsection{论文引用关系}
\subsection{读者行为}
\section{数据库设计}
\section{模块设计}
\section{接口设计}
\section{some troubles}
\end{document} 